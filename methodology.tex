\chapter{Proposed methodology}

To start with, data is need in the form on CAD 3D models.
A 3D visualization program such as SketchUp~\cite{Trimble2014} can be used to open, visualize and edit the models.
Once we have this input we can begin the process to generate \textit{How things work} illustrations from them.

%How are we actually going to solve the problem.
%What is the proposed approach?
%Here we talk about what we are going to use.
%Still not clear what are the actual contributions of the proposal.

\section{Part detection}

As discussed in Sections \ref{ch:intro} and \ref{ch:prevWork}, the first step is to segment the 3D CAD model mesh into useful parts for our task.
In Section \ref{sec:partAnalysis} we presented several options to segment meshes.
However, slippage analysis is the only method that can be applied directly to mechanical machinery.
Moreover, this method gives the degrees of freedom of each part. Therefore, a geometric analysis can be perform and we can work with models that do not have extra time information embedded.
Which is the case for most of the models available in the model libraries in Section \ref{sec:partAnalysis}.
However, none of the methods will automatically detect which parts will interact with the hydraulic fluid.
A first approach will be to manually mark them.
Nevertheless, other paths can be explored, such as heuristics that look at segment shape.
For example, the biggest cylindrical part will probably be the reservoir for the fluid, and the parts connected to it have high probability to have the hydraulic liquid flowing through then.

%How are we going to detect the parts.
%Heuristics to detect were the liquid is?
%The slippage thing, with manual input when needed.
%I.e. to specify pump input or to say where the liquid is.

\section{Fluid simulation}

Once we have segmented the model and we know how each part can move and where can the hydraulic fluid flow, we need to calculate how the liquid interacts with the solid parts.
Given the characteristics of our problem, a physically accurate simulation with two way solid-fluid coupling is needed.
However, extra realistic effects, such as foams or droplets, are irrelevant for this proposal.
Fourier methods are discarded because we we will not be modelling vast amounts of fluid.
Then, we are left we either SPH, grid and hybrid simulations, SPH have become quite popular in the last years.
The great disadvantage of grid simulations is the need to compute the simulation in highly divided grids.
Hybrids systems are more complicated to implement than non hybrid ones.
Nevertheless, for hydraulic machinery the fluid will be constrained to a small area.
Therefore, the previous grid disadvantage is no longer applicable, making it the more appropriate type of simulation for our purposes.

%Using SPH, Grid or Hybrid simulation.
%Looks like grid since we are constrained to a small area and we do care about foams or other extra realism stuff.

\section{Flow visualization}

We are going to use stream lines on surfaces and then replace them by arrows.

\section{Fixed parts visualization}

Rip off gears papers

\section{Animation}

Simulation data is already the animation

\section{Key frames visualization}

sample the animation
For uniform translation only one in between, sample the animation at critical points

\section{Time line}

Table with a time line, diagram?