\documentclass[11pt]{report}
% packages
% Fran Burstall's Bath thesis package
\usepackage{baththesis}
\usepackage{amssymb} %for Blackboard bold etc \usepackage{graphicx} %for including eps graphics % front matter
\usepackage[pdftex]{graphicx}
\usepackage{caption}
\usepackage{subcaption}
\usepackage{url}
\usepackage{amsmath}
\usepackage{txfonts}

\title{ Illustrating how hydraulic machinery works } \author{Garoe Dorta-Perez}
%\degree{Doctor of Philosophy}
\degree{ Master of Science }
\department{Department of Computer Sciences} \degreemonthyear{January 2015}
\norestrictions

\begin{document}
\maketitle
\begin{abstract}
In this research proposal we present a method for automatic depiction of how it works illustrations of hydraulic machinery. \end{abstract}


\chapter{Introduction}
\label{ch:intro}

%1. Introduction to how things work visualization
\textit{How things work} visualizations have been used as an efficient method to explain how a wide range of systems work.
This technique usually involves displaying where each part is in relation to the system, showing how force is transmitted from one piece to the next or animating motion. 
In order to perform this task a range of visual transformations are used.
Such as viewing the system from different angles, zoom degrees,  transparency levels, as well as displaying only a subset of the parts.
Generating material of this sort typically involves manual methods, usually in the form of an expert drawing each illustration by hand or composing a fixed animation using specific software.\\

\begin{figure}
        \centering
        \begin{subfigure}[b]{0.3\textwidth}
                \includegraphics[width=\textwidth]{images/h_pump}
                \caption{Cross section of a typical hydraulic cylinder.}
                \label{fig:h_pump}
        \end{subfigure}%
        ~ %add desired spacing between images, e. g. ~, \quad, \qquad, \hfill etc.
          %(or a blank line to force the subfigure onto a new line)
        \begin{subfigure}[b]{0.3\textwidth}
                \includegraphics[width=\textwidth]{images/piston}
                \caption{A 3D model of a hydraulic piston.}
                \label{fig:pistonModel}
        \end{subfigure}
        ~ %add desired spacing between images, e. g. ~, \quad, \qquad, \hfill etc.
          %(or a blank line to force the subfigure onto a new line)
        \begin{subfigure}[b]{0.3\textwidth}
                \includegraphics[width=\textwidth]{images/cylinder_animation}
                \caption{Frame of manually generated animation of a hydraulic pump.}
                \label{fig:cylinder_animation}
        \end{subfigure}
        \caption{Overview of the intented workflow}\label{fig:AnimOverview}
\end{figure}

%2. Introduction to hydraulics mechanisms
Hydraulic machinery is commonly used in our everyday lives.
Such as lifting cars with jacks, rams on excavators or gerotors to control fuel intake, as shown in Figure~\ref{fig:h_pump}.
Their popularity is based on their faculty to transmit a force or torque multiplication independently of the distance between input and output.  
A typical hydraulic equipment has a contained liquid fluid that becomes pressurised when a force is applied to it.
Then that force is transmitted to the other end of the fluid. 
In order to grasp how the whole system works, it is essential to understand how the pressure is directed and how it interacts with other parts in the machinery. 
Therefore, to illustrate the general process, the spatial configuration of each part in the system must be unveiled.
As well as the chain of motions that takes place within the gears and the liquid fluids.\\

%3. What are the visualization/illustration techniques for hydraulics mechanisms, refer to SIG10 gear paper.
In order to to generate \textit{How things work} visualizations, we first need a 3D model of the physical machinery we will be working with, as shown in Figure~\ref{fig:pistonModel}.
There are some common visualisation and illustration techniques used for hydraulic machinery, as shown in Figure~\ref{fig:cylinder_animation}.
\textbf{Motion arrows} can point out how the solid parts move and they also indicate fluid flow movement.
\textbf{Frame sequences} display key frames in complex motions and they can highlight temporal interactions between the parts.
\textbf{Animations} are an useful tool in highly dynamic systems, for example when an excessive number of frame sequences are needed in a particularly complicated motion.\\

%4. Why this is difficult (designer need to understand force, flow movement, cannot change viewpoint, etc.)
Generating visualisations and illustrations for hydraulic machinery is challenging for designers.
They must understand in detail what forces are generated as parts interact among each other and what kind of flow movements are entailed.
Furthermore, when 2D illustrations are used, they main disadvantage is the impossibility to change the viewpoint to explore the object from different angle.
Moreover, when animations for hydraulic visualizations are generated, they are infrequently updated. 
Since manual animation is a costly and time consuming task.\\

%5. Current status in the research field (mechanical assemblies SIG10, flow systems SIG ASIA 11, maybe some others? No need reference, just some general  discussion, since we have related work part also. And these work cannot handle hydraulics system)
There has been some work done on automatically generating illustrations and visualizations on mechanical assemblies.
Nevertheless, it has been restricted to gear to gear interaction only.
I.e. solid parts interacting with other solid parts, as gear movement can be inferred using symmetry information.
Whereas work on fluid simulation and visualization has not been applied to hydraulic equipment.
As simulations is this field are usually designed to, either generate complex visualizations for engineering purposes, or to produce visually plausible but not physically accurate results for animation or games.\\

%6. Our aim (analysis + visualization, with some description, refer to SIG10 gear paper)
This research proposal aims to introduce a method for generating \textit{How things work} illustrations for hydraulic machinery.
Simulations would be used to generate explanatory illustrations for hydraulic machinery.
This illustrations would help users understanding how this kind of equipment works.\\

%7. Summarize our contribution
In summary, the main contributions would be:
\begin{itemize}
\item An application for creating how things works illustration for hydraulic machinery 3D models.
\item A method for detecting motion and interaction of fluid in the model parts.
\item Algorithms to automatically generate illustrations with motion arrows and frame sequences
\end{itemize}

%Reference example
% Chapter \ref{ch:intro}

\chapter{The Problem}
\label{sec:problem}

Given a 3D CAD model of some hydraulic machinery we want to generate how things work visualizations.
Namely, adding arrows depicting the fluid movement.\\

The problem can be subdivided into:
\begin{enumerate}
\item \textbf{Part analysis:} Detecting fluid containers and fluid handling parts.
\item \textbf{Fluid simulation:} Simulate how the fluid behaves in the previously detected parts.
\item \textbf{Fluid visualization:} Display the fluid simulation data in a intuitive format.\\
\end{enumerate}

Part analysis involves detecting the part type, how it moves and interacts with others.
So the information saved for type would be gear, cylinder, valve, reservoir, etc.
In this section there is a clear difference between the parts that interact with fluids and the ones that do not. 
With respect to types of movement, it would be direction of movement, axis of rotation, axis of translation, etc.\\

Once the parts have been categorized and given an input force, we will have to simulate how the force is transmitted along the different parts.
In the special case where a part is a container of a fluid or is in direct contact with one, that force will have to be introduced in a fluid simulation algorithm.
The output of the simulation will then carry the information along to the next parts.\\

Lastly, in order to visualize the fluid simulation data we will need to generate a visual cue that will indicate intuitively the fluid movement.
A simple approach would be to place arrows indicating the overall fluid movement.

\section{Previous Work}

This proposal is based on the following three areas of previous work.

\subsection{Explanatory illustration}

Explanatory illustration has to adequately transmit motion on a still image, consequently transforming from the temporal space to the image domain.
This is usually found in comics books or in instructions sets.\\

Nienhaus proposed a technique to depict motion in 3D animations~\cite{Nienhaus2005}.
Scene and behaviour descriptions from specialized scene graphs were analysed in order to create the motion cues.  
Researchers have have look into generate automatic illustrations for mechanical assemblies \cite{Mitra2010}.
Furthermore, Lowe showed that even though animations have become a generalized tool for visualizing dynamic systems, special care have to be taken as users can fail to extract the necessary information due to the nature of the animation~\cite{Lowe2014}.
%TODO Extend even more????


\subsection{Fluid simulation}
\label{prevWorkFluidSim}

The current paradigm in fluid simulation consist of solving the Navier-Stokes equations of fluid dynamics, shown in Equations~\ref{eq:navierStokes1} and  ~\ref{eq:navierStokes1}.

\begin{gather}
\label{eq:navierStokes1}
\nabla \cdot \mathbf{u} = 0\\
\label{eq:navierStokes2}
\mathbf{u}_t = -(\mathbf{u} \cdot \nabla)\mathbf{u} + \nabla \cdot ( v \nabla \mathbf{u} - \frac{1}{\rho} \nabla p + \mathbf{f} )
\end{gather}

The first equation enforces mass conservation and ensures incompressibility, while the second encodes momentum conservation it is derived from Newton's Second Law.
Where $\mathbf{u}_t$ is the time derivative vector field of the fluid velocity, $p$  is the scalar pressure field, $\rho$ is the density of the fluid, $v$ is the kinematic viscosity and $\mathbf{f}$ represents the body force per unit mass, usually gravity.\\

Along the years several methods have been proposed that are based on the same Navier-Stokes equations.

\begin{itemize}
\item \textbf{Fourier Transform} techniques uses several superimposed sinusoidal waves to model the fluid behaviour.
\item \textbf{Grid} based methods track the fluid features at fixed points in space.
\item \textbf{Smooth Particle Hydrodynamics} track a large number of particle in the fluid.
\item \textbf{Hybrid} algorithms are combinations of the aforementioned methods.\\
\end{itemize}

Regardless of the chosen method, to update the particles in the for the next frame must be computed.
However, it is common for the algorithms to have a upper time step, if the update is computed after the limit there is no guaranty that the output would be reasonable.\\

Another important concept in fluid simulation is whether the simulation implements interplay within the fluid and rigid bodies that come in contact with it (solid-fluid coupling).
And how flexible is this interaction, one way solid fluid(e.g. a rock drops on a small pool of water, so the water moves but the rock is almost not affected by the  and moves it, but the fluid does not affect the rock), or fluid solid(e.g. a small buoy floating in the ocean has little effect on the surrounding water), and two way solid fluid interaction(e.g. a flexible object drops into a pool of fluid).\\

For more information on real time fluid simulations see Vines survey ~\cite{Vines2012}.
While for survey specific to SPH fluid simulation see Ihmsen~\cite{Ihmsen2014}.

\subsubsection{Fourier Transform}

When simulating fluid with periodic boundary conditions, procedural simulation can be applied with a low computational cost.
Usually this method is used with large masses of water (ocean simulation), as they are an approximation to periodic boundary conditions fluids.
The waves can be modelled as a superimposition of sinusoidal waves, which can be efficiently decomposed using Fast Fourier Transform methods, INSERT FFT CITATION.
Waves produced in this fashion are visually plausible but physical accuracy is not enforced inherently by the model.\\

Each wave has a wave number $k$ and a wave vector $\mathbf{k}$.
Tessendorf \cite{Tessendorf2001} presented the basic technique in this area.
The authors modelled the ocean surface as a summation of complex sinusoids with different wave vectors.
Visually pleasing results were achieved generating random Fourier amplitudes.
The previous method was further improved by Cieutat \cite{Cieutat2003}, with the addition of solid (ships).
And by Chiu \cite{Chiu2006} with adaptive surface tessellation.


ADD MORE AND MOVE SOLID FLUID COUPLING PAPERS TO HERE
%Talk about general simulation and fluid coupling in each of them

\subsubsection{Grid}

Grid methods were among the first techniques to solve fluid simulation and so the firsts ones implement intuitive solutions.
Grid based methods solve Navier-Stokes equations~\ref{eq:navierStokes1} and~\ref{eq:navierStokes2}, in a fixed position in space (i.e. grid).
As fluid flows, the equations are solved in each position, giving a value for speed and pressure at each time step.
This approach is advantageous as many numerical methods can be applied easily on grids and with they are more easily adaptable to GPUs implementations than other methods.\\

One of the firsts papers in this area introduced a Grid method \cite{Foster1996} to solve Navier-Stokes equations~\ref{eq:navierStokes1} and~\ref{eq:navierStokes2} by applying forward Euler time integration. 
Stam \cite{Stam1999} extended this method in order to overcome stability issues.\\

Carlson \cite{Carlson2004} proposed solid-fluid coupling algorithm for grids models using distributed Lagrange multipliers.

\subsubsection{Smooth Particle Hydrodynamics}

Smooth Particle Hydrodynamics(SPH) has been established as one the major breakthroughs for fluid simulations in computer graphics, positioning itself as the most popular method nowadays.
Compared to grid based methods that need high resolution grids to produce foam and splashes, SPH can achieve the same with less computational load.
Two extra advantages are that mass conservation is automatically satisfied if the amount and mass of the particles is keep constant;
and as the particles move with the fluid, the advection term does not need to be calculated explicitly.\\

SPH were first introduced by Desbrun \cite{DesbrunMathieuandGascuel1996}, where each particle encodes its position, velocity, mass and the forces that act on it.\\

While, Muller \cite{Muller2004} presented a method for computing solid fluid coupling. Where boundary particles are created on the surface of the solid and then particle coupling is calculated.
Akinci \cite{Akinci2012} further improved the previous method with the inclusion of irregular particle distributions as well as friction and dragging. 
Shao \cite{Shao2014} also solved stability issues in the previous SPH solid-fluid coupling techniques, using a correction scheme.
Furthermore their algorithm was implemented on GPUs, thus achieving better performance than previous methods.  

\subsubsection{Hybrid}

Hybrid methods are combinations of the aforementioned 


\subsection{Flow visualization}

%In each of them talk first about steady flow and then about unsteady flow 

Extensive work has been done in this area as visualizing fluid movement has a broad range of applications.
However this is a challenging task as it has to effectively display complex and copious amounts of data.
Seeing that fluid simulation is generally solved by means of highly divided grids or with large number of particles, as explained in section~\ref{prevWorkFluidSim}.\\

\begin{figure}
	\centering
	\begin{minipage}[t]{.45\textwidth}
		\centering
		\includegraphics[width=.8\textwidth,height=4cm]{images/streamLinesSpencer}
		\caption{Streamlines on a 3D surface~\cite{Spencer2009}.}
		\label{fig:streamLines}
	\end{minipage}\hfill
	\begin{minipage}[t]{.45\textwidth}
		\centering
		\includegraphics[width=.8\textwidth,height=4cm]{images/streamArrows}
		\caption{Arrows placed on streamlines paths in a 3D surface~\cite{loffelmann1998}.}
		\label{fig:streamArrows}
	\end{minipage}
\end{figure}

Streamlines are convenient tools to describe and visualize flow, as shown in figure~\ref{fig:streamLines}.
A streamline is defined as a curve that is everywhere tangent to flow field, i.e. it is parallel to the local velocity vector.
Therefore, they provide an intuitive mechanism to show the fluid travel direction.
Furthermore, they have properties such as: streamlines will not cross each other (flow will not go across them), or a particle in the fluid starting on one streamline will not leave it.
Once the streamlines has been calculated, instead of displaying them as such, they can be replaced by arrows arranged using some criteria.
For instance, on the path of each streamline or after clustering some streamlines together, as shown in figure~\ref{fig:streamArrows}.\\

For more information on flow visualization see McLoughlin survey on the topic~\cite{McLoughlin2010}.

\subsubsection{2D visualization}

On the 2D image domain, an image-guided algorithm for visualizing 2D flow in images was proposed by Turk \cite{Turk1996}.
Which Li improved, by only generating the fewest possible number of streamlines \cite{Li2008}. 

\subsubsection{Stream lines on surfaces}

Seeding techniques for curves on 3D surfaces were explored by Wicke \cite{Wicke2009}, who developed a technique to combine model reduction with grid based methods.
And by Spencer \cite{Spencer2009}, whose method generates streamlines only for visible parts of the surface, thus providing a significant gain in efficiency.

\subsubsection{Particle tracing lines on surfaces}

\subsubsection{Volume integral objects}

%Could be useful to talk about streamtubes.

IMPORTANT: Make sure the technical/theoretical/practical issues of previous work is clearly stated 

\chapter{Proposed methodology}

How are we actually going to solve the problem.
What is the proposed approach?

\section{Time line}

Table with a time line, diagram?

\chapter{Evaluation}

we plan to test it on a variety of models, other way is comparison with previous work, user study -> like doing and study of how well they understood how the machinery works, … try think about more ideas to test the approach, comparison of frame sequence, animation and arrows, which one does the user think is more useful 

\chapter{Applications}

generating visualizations for museums, schools, education in general

\chapter{Conclusions}



\bibliographystyle{eg-alpha-doi}

\bibliography{baththesis}


\end{document}
