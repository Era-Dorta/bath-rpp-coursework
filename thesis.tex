\documentclass[11pt]{report}
% packages
% Fran Burstall's Bath thesis package
\usepackage{baththesis}
\usepackage{amssymb} %for Blackboard bold etc \usepackage{graphicx} %for including eps graphics % front matter
\usepackage[pdftex]{graphicx}
\usepackage{url}

\title{ Illustrating how hydraulic machinery works } \author{Garoe Dorta-Perez}
%\degree{Doctor of Philosophy}
\degree{ Master of Science }
\department{Department of Computer Sciences} \degreemonthyear{January 2015}
\norestrictions

\begin{document}
\maketitle
\begin{abstract}
In this research proposal we present a method for automatic depiction of how it works illustrations of hydraulic machinery. \end{abstract}


\chapter{Introduction}
\label{ch:intro}

\textit{How things work} visualizations have been used as an efficient method to explain how a wide range of systems work. 
This technique usually involves displaying where each part is in relation to the system, showing how force is transmitted from one piece to the next and animating motion. 
In order to achieve this, a range of visual transformations are used, such as viewing the system from different angles, zoom degrees,  transparency levels, as well as displaying only a subset of the parts.

In the field of hydraulics, simulations can be used to illustrate how hydraulic machinery works.
A common hydraulic equipment has  some type of liquid fluid that becomes pressurised when a force is applied to it and then that force is transmitted to the other end of the fluid. 
Understanding how the pressured is directed and how it interacts with other parts in the machinery is essential in order to grasp how the whole system works. 
Therefore in order to illustrate the general process the user needs to know the spatial configuration of all the parts in the system and the chain of motions that takes place within the gears and the liquid fluids as well.

This research proposal aims to introduce an automatic method for generating \textit{How things work} illustrations for hydraulic machinery. 

In summary, the main contributions would be:
\begin{itemize}
\item An application for creating how things works illustration for hydraulic machinery 3D models.
\item A method for detecting motion and interaction of fluid inside 3D model parts.
\item Algorithms to automatically generate illustrations with motion arrows and frame sequences
\end{itemize}

%Reference example
% Chapter \ref{ch:intro}

\section{The Problem}
\label{sec:problem}

Given a 3D CAD model of some hydraulic machinery we want to generate how things work visualizations.
Namely, adding arrows depicting the fluid movement.

The problem can be subdivided into:
\begin{enumerate}
\item Part analysis: Detecting fluid containers and fluid handling parts.
\item Fluid simulation: Simulate how the fluid behaves in the previously detected parts.
\item Fluid visualization: Generate arrows to show the results of the simulation. 
\end{enumerate}

\section{Previous Work}

This proposal is based on the following three main areas of previous work.

\subsection{Explanatory illustration}

Explanatory illustration has been effectively used to show complex and/or copious amounts of  scientific and technical data.
Researchers have have look into generate automatic illustrations for mechanical assemblies \cite{Mitra2010}.
%TODO Improve and extend this, but it is difficult to find literature on this topic


\subsection{Fluid simulation}

Fluid simulation is a well known research area. 
One of the firsts papers in this area introduced a Grid method \cite{Foster1996} to solve the Navier-Stokes equations by applying forward Euler time integration. 
Stam \cite{Stam1999} extended the method in order to overcome stability issues. 
More recent simulations introduced the Smooth Particle Hydrodynamics (SPH) technique \cite{DesbrunMathieuandGascuel1996}. 
This simulations did not include including solid-fluid coupling, as it was proposed in grids models by Carlson \cite{Carlson2004}. 
On the other hand, Muller \cite{Muller2004} did the same for SPH simulations, which Akinci \cite{Akinci2012} further improveded with the inclusion of friction and dragging. 
Shao \cite{Shao2014} also solved stability issues in the previous SPH solid-fluid coupling techniques.

\subsection{Flow visualization}

Streamlines are the standard approach to produce flow visualization. 
Seeding techniques for curves on 3D surfaces were explored by Wicke \cite{Wicke2009} and Spencer \cite{Spencer2009}.

\chapter{Data Structures Used in this research}
\section{The 4D-Stack - A Revolutionary Data Structure}
The 4D-Stack turned out to be a complete disaster as traversal time approached $O(n^9).$
It is best illustrated by the following equation: $F(x) = \prod_{0\leq i<k}d_i(x)$
but the following may not be true:

\[
-f(x) = - \log \prod_{0\leq i<k}d_i(x) = - \sum_{0\leq i<k} \log d_i(x)
\]


\section{More Irrelevant Stuff}
If you want to put numbers on equations use this form:
\begin{equation}
\label{integralrep}
f(x)=\int_0^L h( \langle x-p(t),n(t) \rangle ) dt\,.
\end{equation}



\chapter{Results}
\begin{table}[tbp]
\begin{tabular}{||l|l|l|p{1.5in}|p{1.2in}||}
\hline
\hline
Name & Dates & Degree & Title & Present Position \\
\hline
Rudolphe Neyrouge & 1953  - & PhD &  Non-linear Fictional Analysis &  co-tutelle with NPole University, Arctic \\
Rip van Winkle & 1754 -  & PhD & Modelling 4D Harmonic Maps & University of Old People \\

\hline
{\bf Graduated} &&&& \\
\hline
Johnny Depp & 1967 - 2009 & MSc & How to Act an MSc & Actor \\
Valentina Lsitsa & 1992 - 2013 & MSc & Hitting the right notes &  Pianist  \\
Johann S. Bach & 1567 - 1953 & MSc & Interactive Piano  & Composer \\\hline
\hline
\end{tabular}
\caption{\label{stud-table} Graduate Students, last seven years.}
\end{table}

Due to a time quake during the research, the results were catapulted into the future.  They will appear in about 20 years.
In the meantime to demonstrate the use of tables, please see table~\ref{stud-table} for a list of students who took more than 30 years to graduate.

\chapter{Conclusions and Future Work}
No conclusions can be drawn until the results appear and no future work is recommended.

\bibliographystyle{eg-alpha-doi}

\bibliography{baththesis}


\end{document}
