\chapter{Evaluation}

The system will be tested by generating illustrations for a wide variety of models.
Since we do not know of any previous work which has generated \textit{How things work} illustrations for hydraulic machinery, comparison with other methods is limited to contrasting our work against manually generated illustrations.
Being able to replicate a series of chosen hydraulic illustrations could be a goal of the project. 
Another path to evaluate the effectiveness of the generated data is to perform user studies.
One example could be an evaluation on how well the test subjects understood the machinery internal workings with a series of questions, to estimate their knowledge on the equipment before and after seeing the illustrations.
Another could measure which illustration the user finds as more instructing, the automatically generated ones or manually generated, as well as asking them why they think in such a manner.
Moreover, user studies can be helpful in fine tuning parameters in the algorithm, therefore improving the overall quality of the results.
Comparing frame sequences, animations and single illustrations is another potential evaluation study.

%we plan to test it on a variety of models, other way is comparison with previous work, user study -> like doing and study of how well they understood how the machinery works, … try think about more ideas to test the approach, comparison of frame sequence, animation and arrows, which one does the user think is more useful symmetry information