\chapter{Applications and Conclusions}

\section{Applications}

We are proposing a tool whose main aim are educational purposes.
The most obvious applications would be the generation of \textit{How things work} visualizations for museums, schools and educational centres in general.
Such institutions would benefit from having a illustration generator tool.
Its semi-automatic nature implies that, it will address the issues with depictions created manually by experts that were discussed in Chapter~\ref{ch:intro}, such as a time consuming generation stage or lack of actualization when new machinery arrives.

\section{Conclusions}

We have presented a proposal for a semi-automatic \textit{How things work} illustrations generator for hydraulic machinery.
The system would work with 3D CAD models of hydraulic assemblies and it would output the illustrations, either single-frame or key frames.
The work flow of the algorithm would include a part analysis stage, were the 3D model is segmented in sections with high level information, such as type or rotational axis, a fluid simulation stage were the hydraulic interactions are calculated and, finally, a visualization stage where techniques from flow visualization and illustrating motion in single images are used to depict the overall internal workings of hydraulic machinery.